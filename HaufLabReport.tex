%\begin{itemize}
%\end{itemize}
%\framebreak
%\begin{figure}
%\centering
%   \def\svgwidth{0.75\columnwidth}
%\input{}
%\caption{\tiny {}}
%\end{figure}
% Copyright 2004 by Till Tantau <tantau@users.sourceforge.net>.
%
% In principle, this file can be redistributed and/or modified under
% the terms of the GNU Public License, version 2.
%
% However, this file is supposed to be a template to be modified
% for your own needs. For this reason, if you use this file as a
% template and not specifically distribute it as part of a another
% package/program, I grant the extra permission to freely copy and
% modify this file as you see fit and even to delete this copyright
% notice. 

\documentclass{beamer}
\usepackage{graphicx}
%\usepackage{enumitem}
\setbeamertemplate{frametitle continuation}{\frametitle{\color{white}Title}}
% There are many different themes available for Beamer. A comprehensive
% list with examples is given here:
% http://deic.uab.es/~iblanes/beamer_gallery/index_by_theme.html
% You can uncomment the themes below if you would like to use a different
% one:
%\usetheme{AnnArbor}
%\usetheme{Antibes}
%\usetheme{Bergen}
%\usetheme{Berkeley}
%\usetheme{Berlin}
%\usetheme{Boadilla}
%\usetheme{boxes}
%\usetheme{CambridgeUS}
%\usetheme{Copenhagen}
%\usetheme{Darmstadt}
%\usetheme{default}
%\usetheme{Frankfurt}
%\usetheme{Goettingen}
%\usetheme{Hannover}
%\usetheme{Ilmenau}
%\usetheme{JuanLesPins}
%\usetheme{Luebeck}
%\usetheme{Madrid}
%\usetheme{Malmoe}
%\usetheme{Marburg}
%\usetheme{Montpellier}
%\usetheme{PaloAlto}
%\usetheme{Pittsburgh}
%\usetheme{Rochester}
\usetheme{Singapore}
%\usetheme{Szeged}
%\usetheme{Warsaw}

\title{Project report}

% A subtitle is optional and this may be deleted
\subtitle{Rotation in the Hauf Lab}

\author{Amogh Jalihal\inst{1}}
% - Give the names in the same order as the appear in the paper.
% - Use the \inst{?} command only if the authors have different
%   affiliation.

\institute[Virginia Tech] % (optional, but mostly needed)
{
  \inst{1}%
  Genetics, Bioinformatics and Computational Biology
}
%  \and
%  \inst{2}%
%  Department of Theoretical Philosophy\\
%  University of Elsewhere}
% - Use the \inst command only if there are several affiliations.
% - Keep it simple, no one is interested in your street address.

\date{9th Dec, 2015}
% - Either use conference name or its abbreviation.
% - Not really informative to the audience, more for people (including
%   yourself) who are reading the slides online

%\subject{Theoretical Computer Science}
% This is only inserted into the PDF information catalog. Can be left
% out. 

% If you have a file called "university-logo-filename.xxx", where xxx
% is a graphic format that can be processed by latex or pdflatex,
% resp., then you can add a logo as follows:

% \pgfdeclareimage[height=0.5cm]{university-logo}{university-logo-filename}
% \logo{\pgfuseimage{university-logo}}

% Delete this, if you do not want the table of contents to pop up at
% the beginning of each subsection:
%\AtBeginSubsection[]
%{
%  \begin{frame}<beamer>{Outline}
%    \tableofcontents[currentsection,currentsubsection]
%  \end{frame}
%}

% Let's get started
\begin{document}

\begin{frame}
  \titlepage
\end{frame}

\begin{frame}{Outline}
  \tableofcontents
  % You might wish to add the option [pausesections]
\end{frame}

% Section and subsections will appear in the presentation overview
% and table of contents.
\section{Background}

\subsection{The Biology}
\begin{frame}{The biology}
\includegraphics[scale=0.5]{cell_cycle.png}
  \begin{itemize} 
\item An overview of the cell cycle 
\begin{itemize}
  \item{    Driving the cell through the division cycle  }
  \item{Regulators and checkpoints}
  \item{   Dynamics of molecular players}
  \end{itemize}
\item But why study it?	

\begin{itemize}
  \item{  Basis of life}

  \item{  Cancer}

  \item{  Growth and development}
\end{itemize}
\end{itemize}
\end{frame}
\subsection{...and the beast}
\begin{frame}{...and the beast}

\begin{itemize}
\item{ \it Schizosaccharomyces pombe }
\begin{center}
\includegraphics{pombe_ascus_SEM.jpg}

\tiny{http://www.hauflab.org/microscopy-images/}
\end{center}
\item Why we use it..
\begin{itemize}
\item Growth and maintenance
\item Strain contruction
\item Live cell imaging
\end{itemize}
\item ..for studying the cell cycle
\begin{itemize}
\uncover \item A single Cdk-1
\item Only 3 chromosomes
\end{itemize}
\end{itemize}

\end{frame}

\section{Two topics}
\subsection{Order in chromosome segregation}
\begin{frame}[allowframebreaks]{Order in chromosome segregation}
\begin{itemize}
	\item Motivation 
		\begin{itemize}
			\item Julia Kamenz's manuscript \cite{JulKam2015}
			\item Possible explanations of order?
			\item Castagnetti et al (2010) report nuclear division in absence of spindle microtubules \cite{Casta2010}

\emph{... during nuclear fission SPBs and sister chromatids separate in absence of spindle microtubules, that {\bf some level of chromosome segregation can take place},... for efficient nuclear fission functional SPBs and sister chromatid separation is required.} [Emphasis mine]
				%\begin{itemize}
				%\item ``Nuclear Fission'' in absence of microtubules
				%\item At $25^{o}C$ mitosis delayed by 45 min
				%\item MBC treatment: short stubs of microtubules ($~ 1 \mu m$)
				%\end{itemize}
		\end{itemize}
\end{itemize}
\end{frame}
\begin{frame}[allowframebreaks]{Story 1: MBC, mitosis, microscopy..}{To verify observations of Castagnetti et al and to study chrososome segregation in ``nuclear fission''}

\begin{itemize}
\item Perils of overeager strain construction:{\it mad3$\Delta$ cen2-lacI-GFP cen3-tetR-tdTomato/dh1L-tetR-tdTomato}
\item Controls
	\begin{itemize}
	\item MBC effect on tubulin
	\item Temperature
	\end{itemize}
\item Control 1: SK399: { \it atb2-GFP plo1-mCherry}
\item Control 2: SU603/SU604: {\it mad3$\Delta$ plo1-mCherry cen2-GFP}
\end{itemize}
\framebreak
\begin{center}
\includegraphics[scale=0.5]{MBC_Control_SK399_1.jpg}

SK399: atb2-GFP plo1-mCherry
\end{center}
\framebreak

\begin{center}
\includegraphics[scale=0.3]{MBC_Control_SK399_2.png}

SK399: atb2-GFP plo1-mCherry
\end{center}

\framebreak

\begin{center}
%\includegraphics[scale=1.5]{su_2.png}
 {\it mad3$\Delta$ plo1-mCherry cen2-GFP}

\begin{tabular}{|l l|}
\hline
cen2 split after observable plo1 split & 10 \\
cen2 split during plo1 window (plo1 split not observed) & 23 \\
cen2 split after plo1 window & 9 \\
non-splitting cen2 in plo1 window & 8 \\
cen2 split in absence of plo1 & 3\\
\hline
Total number observed & 53 \\
\hline
\end{tabular}
\end{center}

\framebreak
\begin{figure}
\centering
\includegraphics[scale=2.2]{kymograph_long.png}
\end{figure}

\begin{figure}
\includegraphics[scale=1.5]{kymo_large_window.pdf}
\end{figure}
\end{frame}
\begin{frame}{Conclusions 1}
\begin{itemize}
\item Observations coincide with that of Castagnetti et al.
\item Practical considerations: 3 color filming?
\item General take-home: What effect are is really being descibed?
\end{itemize}
\end{frame}
\subsection{Temporal dynamics of mitotic regulators}
\begin{frame}{Temporal dynamics of mitotic regulators}
\begin{itemize}	
	\item Motivation
		\begin{itemize}	
			\item Cdc13 and the apparent stabilization
			\item Automating image analysis
			\item Tool to study temporal dynamics from microscopy
		\end{itemize}
		
\end{itemize}
\end{frame}
\begin{frame}[allowframebreaks]{Story 2: ImageJ, automation, insight}{To attempt to create an ImageJ based workflow to automate image analysis to obtain temporal dynamics}
\begin{figure}
\includegraphics{cartoon1.png}
\end{figure}
\framebreak
\begin{figure}
\includegraphics{cartoon2.png}
\end{figure}
\framebreak
\begin{figure}
\includegraphics{cartoon3.png}
\end{figure}
\framebreak
\begin{figure}
\includegraphics{cartoon4.png}
\end{figure}


\framebreak
\begin{itemize}
	\item Workflow
		\begin{enumerate}
		\item IF single cell AND fixed AND no cells enter 
		\item Transform cells $\rightarrow$ STEP 1.
		\item Threshold ROI OR create Mask
		\item ``Analyze Particles''
		\item ``Clean'' output 
		\item Scilab script plots, assembles plot lines.
		\item Registration? Segmentation $\rightarrow$ GOTO STEP 1.
		\end{enumerate}

\end{itemize}
\end{frame}

\framebreak
\begin{figure}
\centering
   \def\svgwidth{0.75\columnwidth}
\input{st957_cdc13_nucred_cellgreen.pdf_tex}
\caption{ST957: {\it cdc13-sfGFP dh1L-tetR-tdTomato}}
\end{figure}

\framebreak

\begin{figure}
\centering
   \def\svgwidth{0.75\columnwidth}
\input{st650_cdc2nucred_cdc2cellgreen_notNor1.pdf_tex}
\caption{ST650: {\it cdc2-GFP dh1L-tetR-tdTomato}}
\end{figure}

\framebreak
\begin{figure}

\centering
   \def\svgwidth{0.75\columnwidth}
\input{sl249_cut2_nucred_cellgreen.pdf_tex}
\caption{SL249: {\it cut2-GFP dh1L-tetR-tdTomato}}
\end{figure}

\framebreak
\begin{figure}
\centering
   \def\svgwidth{0.75\columnwidth}
\input{su076_cut2nucgreen_cdc2nucred.pdf_tex}
\caption{SU076: {\it cdc2-mCherry cut2-GFP}}
\end{figure}


\framebreak
\begin{figure}
\centering
   \def\svgwidth{0.75\columnwidth}
\input{su076_cut2nucgreen_cdc2nucred.pdf_tex}
\caption{SU076: {\it cdc2-mCherry cut2-GFP}}
\end{figure}


\framebreak

\begin{figure}
\centering
   \def\svgwidth{0.75\columnwidth}
\input{st998_20xcdc13nucgreen_cdc2nucred.pdf_tex}
\caption{ {ST998: {\it 20X-cdc13-sfGFP cdc2-mCherry}}}
\end{figure}

\framebreak
\begin{figure}
\centering
   \def\svgwidth{0.75\columnwidth}
\input{su417_cdc13_wcgreen_nucred.pdf_tex}
\caption{SU417: {\it plo1-mCherry 20Xcdc13-sfGFP}}
\end{figure}

\framebreak
\begin{figure}
\centering
   \def\svgwidth{0.75\columnwidth}
\input{su302_nucgreen_nucred_declines_dont_match.pdf_tex}
\caption {SU302: {\it 8Xcut2-GFP cdc2-mCherry}}
\end{figure}

\framebreak
\begin{figure}
\centering
   \def\svgwidth{0.75\columnwidth}
\input{su303_cut2wr_cut2ng.pdf_tex}
\caption{SU303: {\it 8Xcut2-GFP cdc2-mCherry }}
\end{figure}

\begin{frame}{Conclusions 2}
\begin{itemize}
\item Cdc13 and Cdc2 seem to exhibit preferential export from the nucleus on entry into mitosis
\item Overexpressing cut2 leads to whole cell degradation?
\end{itemize}
\end{frame}

\section{Summary}

\begin{frame}{Summary}
\begin{itemize}
	\item MBC treatment experiment inconclusive
	\item Cdc13, cdc2, cut2 dynamics studied
\end{itemize}
\end{frame}

\section{Outlook}
\begin{frame}{Outlook}
\begin{itemize}
	\item Other approaches to study centromere separation
	\item 

\end{itemize}
\end{frame}
% You can reveal the parts of a slide one at a time
% with the \pause command:
%\begin{frame}{Second Slide Title}
%  \begin{itemize}
%  \item {
%    First item.
 %   \pause % The slide will pause after showing the first item
 
%  \item {   
%    Second item.%
 % }
%  % You can also specify when the content should appear
%  % by using <n->:
%  \item<3-> {
%    Third item.
%  }
%  \item<4-> {
%    Fourth item.
%  }
  % or you can use the \uncover command to reveal general
%  % content (not just \items):
%  \item<5-> {
%    Fifth item. \uncover<6->{Extra text in the fifth item.}
%  }
%  \end{itemize}
%\end{frame}

% \section{Second Main Section}
% 
% \subsection{Another Subsection}
% 
% \begin{frame}{Blocks}
% \begin{block}{Block Title}
% You can also highlight sections of your presentation in a block, with it's own title
% \end{block}
% \begin{theorem}
% There are separate environments for theorems, examples, definitions and proofs.
% \end{theorem}
% \begin{example}
% Here is an example of an example block.
% \end{example}
% \end{frame}
% 
% % Placing a * after \section means it will not show in the
% % outline or table of contents.
% \section*{Summary}
% 
% \begin{frame}{Summary}
%   \begin{itemize}
%   \item
%     The \alert{first main message} of your talk in one or two lines.
%   \item
%     The \alert{second main message} of your talk in one or two lines.
%   \item
%     Perhaps a \alert{third message}, but not more than that.
%   \end{itemize}
%   
%   \begin{itemize}
%   \item
%     Outlook
%     \begin{itemize}
%     \item
%       Something you haven't solved.
%     \item
%       Something else you haven't solved.
%     \end{itemize}
%   \end{itemize}
% \end{frame}
% 
% 
% 
% % All of the following is optional and typically not needed. 
% \appendix
% \section<presentation>*{\appendixname}
% \subsection<presentation>*{For Further Reading}
% 
 \begin{frame}[allowframebreaks]
%   \frametitle<presentation>{For Further Reading}
%     
  \begin{thebibliography}{10}
%     
%   \beamertemplatebookbibitems
%   % Start with overview books.
% 
	
	\bibitem{JulKam2015}
	Kamenz et al.~
	\newblock {\em Synchronous sister chromatid splitting in anaphase occurs without obligatory positive feedback}

	\bibitem{Casta2010}
     Castagnetti et al.~
     \newblock {\em Fission yeast cells undergo nuclear division in the absence of spindle microtubules}
     \newblock PLOS Biology, 2010.
	
%  
%     
%   \beamertemplatearticlebibitems
%   % Followed by interesting articles. Keep the list short. 
% 
%   \bibitem{Someone2000}
%     S.~Someone.
%     \newblock On this and that.
%     \newblock {\em Journal of This and That}, 2(1):50--100,
%     2000.
   \end{thebibliography}
\end{frame}

\begin{frame}{Thank you}

\end{frame}

\end{document}


